% class
\documentclass[a4paper,12pt,xelatex,ja=standard]{bxjsarticle}

% packages
%% mathematical notations
\usepackage{amsthm,amsmath,amssymb,amsfonts} % mathematical notations
\usepackage{bm} % bold character
\usepackage{latexsym} % more mathematical notations
\usepackage{physics} % physical notations
%% graphs
\usepackage[dvipdfmx]{graphicx, xcolor} % graph
\usepackage{circuitikz} % for circuit elements
\usepackage{float} % positioning of graphs
\usepackage{siunitx} % SI units
\usepackage{tikz} % graphic elements
\usepackage{wrapfig} % must be after float package.
%% type system
\usepackage{bussproofs} % proof tree
%% code
\usepackage[ruled,vlined]{algorithm2e} % pseudo code
\usepackage{listings} % source code

% Basic information
\title{電子情報学専攻 \, 専門 \\ 令和元年 \, 解答・解説}
\author{diohabara}
\date{\today}

\begin{document}
\maketitle

\section*{第1問\ 電気・電子回路}

\subsection*{(1)}

% ref: https://jeea.or.jp/course/contents/01133/

スイッチを短絡させると、以下の回路方程式が成り立つ。

\[
  L \frac{d i(\tau)}{d \tau} = v(t)
\]

電源は定電圧電源なので \(v(t) = v(0) = E\) であり、 \(\tau = [0, t]\) 上で \(\tau\) に関して積分して整理すると

\[
  i(t) - i(0) = \frac{E}{L}t
\]

問題文より $t = 0$ で $i(0) = 0$ だから

\[
  i(t) = \frac{E}{L}t
\]

\subsection*{(2)}

スイッチを開放すると、 \(t = [T_0, T_0 + T_1)\) において以下の回路方程式が成り立つ。

\[
  L \frac{d i(t)}{d t} + \frac{1}{C}\int^{t}_{T_0}i(\tau) d \tau = E
\]

\section*{第2問\ 論理回路}
  \subsection*{(1)}
  \subsection*{(2)}
  \subsection*{(3)}
  \subsection*{(4)}
  \subsection*{(5)}

\section*{第3問\ アルゴリズム}
  \subsection*{(1)}
  \subsection*{(2)}
  \subsection*{(3)}
  \subsection*{(4)}

\section*{第4問\ ネットワーク}
  \subsection*{(1)}
  \subsection*{(2)}
  \subsection*{(3)}
  \subsection*{(4)}
  \subsection*{(5)}
    \subsubsection*{(a)}
    \subsubsection*{(b)}

\section*{第5問\ 情報理論}
\subsection*{(1)}
\subsection*{(2)}
\subsection*{(3)}
\subsection*{(4)}
\subsection*{(5)}
\subsection*{(6)}
\subsection*{(7)}

\end{document}

