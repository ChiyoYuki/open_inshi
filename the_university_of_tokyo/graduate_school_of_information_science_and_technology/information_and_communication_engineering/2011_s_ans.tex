% class
\documentclass[a4paper,12pt,xelatex,ja=standard]{bxjsarticle}

% packages
%% mathematical notations
\usepackage{amsthm,amsmath,amssymb,amsfonts} % mathematical notations
\usepackage{bm} % bold character
\usepackage{latexsym} % more mathematical notations
\usepackage{physics} % physical notations
%% graphs
\usepackage{graphicx, xcolor} % graph
\usepackage{circuitikz} % for circuit elements
\usepackage{float} % positioning of graphs
\usepackage{siunitx} % SI units
\usepackage{tikz} % graphic elements
\usepackage{wrapfig} % must be after float package.
%% type system
\usepackage{bussproofs} % proof tree
%% code
\usepackage[ruled,vlined]{algorithm2e} % pseudo code
\usepackage{listings} % source code
\usepackage{inconsolata}
\lstset{
  basicstyle=\footnotesize,
  numbers=left,
  frame={tb}
}

% Basic information
\title{電子情報学専攻 \, 専門 \\ 平成22年 \, 解答・解説}
\author{diohabara}
\date{\today}

\begin{document}
\maketitle

\section*{第1問\ 電気・電子回路}

\section*{第2問\ 計算機アーキテクチャ}
\subsection*{(1)}
\subsubsection*{初期参照ミス}
キャッシュラインを最初にアクセスしたときに起こるミス

\subsubsection*{競合性ミス}
同じインデックスを持つ異なるキャッシュラインにアクセスすることで起こるミス

\subsubsection*{容量性ミス}
キャッシュしたいライン数がキャッシュの容量を上回ることで起こるミス

\subsection*{(2)}
競合性ミス
(フルアソシアティブキャッシュではインデックスを用いず、参照されるごとにすべてのラインのタグとメモリアドレス内のタグを比較するため。)

\subsection*{(3)}
1つのインデックスに対応するキャッシュラインの集合、セットの大きさを大きくする。つまり、連想度を大きくする。

\subsection*{(4)}
\begin{itemize}
  \item キャッシュは新しく領域を確保する
  \item メモリからデータを取ってきて領域に格納する
\end{itemize}

\subsection*{(5)}
TLB、Translation Lookaside Bufferとは仮想アドレスと物理アドレスの変換表であるページテーブル専用のフルアソシアティブ方式のキャッシュである。仮想ページアドレスをタグとし、対応する物理ページアドレスを格納している。

\subsection*{(6)}
ページサイズが$4096$バイトなので、$\log 4096 = 12$ビットのページ内オフセットが必要である。\\
仮想アドレスが$32$ビットなので仮想ページアドレスは$32 - 12 = 20$ビットとなる。これがTLBのタグとなる。\\
同様に物理アドレスが$31$なので物理ページアドレスは$31 - 12 = 19$ビットとなる。となる。これがTLBのキャッシュラインの大きさである。\\
更に有効ビット1ビットを考慮して、TLBは64エントリなので、求めるTLBの大きさは
\[
  (1 + 20 + 19) \times 64 = 2560\text{[bit]} = 320\text{[byte]}
\]

\subsection*{(7)}
\begin{enumerate}
  \item ページテーブルを参照する
  \item TLBに空いているエントリがある場合、現在参照している仮想ページアドレスに対応する物理ページアドレスが入れられる
  \item TLBが空いていない場合、LRUラインなエントリを使って入れる
\end{enumerate}

\subsection*{(8)}
ページフォールト

\subsection*{(9)}
\begin{itemize}
  \item CPUの処理を一時中断する
  \item 例外を起こして処理をOSに移し、OSの特権的命令を用いる
  \item テーブルのエントリに入っている二次記憶上のアドレスから主記憶の空いている場所にページをコピーする
  \item ページテーブルのエントリに物理ページアドレスを書き込み、有効ビットを1にする
  \item 主記憶に空いた場所がなければどれか1つのページを二次記憶上に追い出して、空いた場所に必要とされるページを読み込む
\end{itemize}

\subsection*{(10)}
$\log 4096 = 12$より、ページ内オフセットは$12$ビットである。これだけがキャッシュアクセスに使える。\\
キャッシュラインが$64$バイトで1ワードが$32$ビットなので、$(64 \times 8) / 32 = 16$より1ラインに$16$ワード入るから、ワードを特定するために$\log 16 = 4$ビットが必要。\\
更に1ワードの中に$4$バイトが入るから、バイトを特定するために$\log 4 = 2$ビットが必要。\\
更にキャッシュアクセスのインデックスとして使われるのは$12 - 4 - 2 = 6$ビットとなる。\\
2ウェイセットアソシアティブであり、1ラインのデータ語が64バイトなので、求めるキャッシュの最大容量(データ語)は
\[
  2^6 \times 2 \times 64 = 8192 \text{[byte]}
\]

\subsection*{(11)}
TLBを参照する時間を待ってからメモリ語にアクセスするので、メモリ語にアクセスされるまでの時間が多くかかる。

\subsection*{(12)}
\subsubsection*{名称}
エイリアス
\subsection*{どのようなものか}
別の2つの仮想アドレスが同じ物理アドレスを指している場合、片方の仮想アドレスによるキャッシュの更新がもう片方の仮想アドレスを利用する側に伝わらないという問題が発生する。

\section*{第3問\ アルゴリズムとデータ構造}
\subsection*{(1)}

\subsection*{(2)}

\subsection*{(3)}

\subsection*{(4)}

\subsection*{(5)}

\section*{第4問\ 情報通信}

\section*{第5問\ 信号処理}
\subsection*{(1)}

\subsection*{(2)}

\subsection*{(3)}

\subsection*{(4)}

\subsection*{(5)}

\end{document}

