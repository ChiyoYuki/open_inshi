% class
\documentclass[a4paper,12pt,xelatex,ja=standard]{bxjsarticle}

% packages
%% mathematical notations
\usepackage{amsthm,amsmath,amssymb,amsfonts} % mathematical notations
\usepackage{bm} % bold character
\usepackage{latexsym} % more mathematical notations
\usepackage{physics} % physical notations
%% graphs
\usepackage{graphicx, xcolor} % graph
\usepackage{circuitikz} % for circuit elements
\usepackage{float} % positioning of graphs
\usepackage{siunitx} % SI units
\usepackage{tikz} % graphic elements
\usepackage{wrapfig} % must be after float package.
%% type system
\usepackage{bussproofs} % proof tree
%% code
\usepackage[ruled,vlined]{algorithm2e} % pseudo code
\usepackage{listings} % source code
\usepackage{inconsolata}
\lstset{
  basicstyle=\footnotesize,
  numbers=left,
  frame={tb}
}

% Basic information
\title{電子情報学専攻 \, 専門 \\ 平成23年 \, 解答・解説}
\author{diohabara}
\date{\today}

\begin{document}
\maketitle

\section*{第1問\ 電気・電子回路}

\section*{第2問\ 計算機アーキテクチャ}
\subsection*{(1)} % https://en.wikipedia.org/wiki/Dependence_analysis
依存は2つの命令で、あるレジスタxに対して、読み出し、書き込みがどの順番で行われているかによって定まる。
\subsubsection*{フロー依存(Read after Write)}
命令Aで書き込んだ値を後続の命令Bで読み出すことで起こるAからBへの依存関係。\\
下の例では1行目で書き込んだレジスタxを2行目で読み出している。
\begin{lstlisting}
  x = 20
  y = x + 12
\end{lstlisting}

\subsubsection*{逆依存(Write after Read)}
命令Aで読みだしたレジスタに後続の命令Bが書き込みを行うことで起こるAからBの依存関係。\\
下の例では1行目で読みだしたレジスタxに2行目で書き込みを加えている。
\begin{lstlisting}
  y = x + 20
  x = 12
\end{lstlisting}

\subsubsection*{入力依存(Read after Read)}
命令Aと命令Bで同じレジスタから読み出しを行うこと依存関係。 \\
下の例では1行目と2行目ともにレジスタxの読み込みをしている。
\begin{lstlisting}
  y = x + 20
  z = x + 12
\end{lstlisting}

\subsubsection*{出力依存(Write after Write)}
命令Aで書き込んだレジスタに後続の命令Bが再度書き込みを行うことで起こるAからBの依存関係。\\
下の例では1行目と2行目ともにレジスタxの書き込みをしている。
\begin{lstlisting}
  x = 20
  x = 12
\end{lstlisting}

\subsection*{(2)}
フロー依存、逆依存、出力依存

\subsection*{(3)}
それぞれ真、偽、偽

\subsection*{(4)}
与えられたアセンブリコードでは、L1、L3, L4でr1に書き込みをし、L2でr2に書き込みをしている。一方、L2、L3、L4でr1を読み込みし、L4でr2を読み込みしている。\\
\subsubsection*{フロー依存(Read after Write)}
フロー依存は命令Aで書き込みをしたレジスタを命令Bで読み込んでいるときに生じる。
このことから組み合わせを考えると
\[
  (A, B) = (L1, L2), (L1, L3), (L1, L4), (L2, L3), (L2, L4), (L2, L4)
\]

\subsubsection*{逆依存(Write after Read)}
フロー依存は命令Aで読み込みをしたレジスタを命令Bで書き込んでいるときに生じる。
このことから組み合わせを考えると
\[
  (A, B) = (L2, L3), (L2, L4), (L3, L4)
\]

\subsection*{(5)}
新しく変数を定義することで偽のデータ依存は解消できる。\\
L3での書き込みレジスタr1とL4での読み込みレジスタr1を新しいレジスタr3とすることで、L2とL3を並列に動作させることが可能となる。

\section*{第3問\ データベース}

\section*{第4問\ 情報通信}

\section*{第5問\ 信号処理}
\subsection*{(1)}
畳み込み積分のフーリエ変換はそれぞれの関数のフーリエ変換の積となるから
\[
  O(\omega) = G(\omega)H(\omega) + N(\omega)
\]

\subsection*{(2)}
\begin{equation*}
  \begin{split}
    G(\omega)
      &= \int^{\infty}_{-\infty} g(t) e^{-j \omega t} dt \\
      &= \int^{\infty}_{-\infty} \left( \sum^{\infty}_{k = - \infty} \frac{1}{T} e^{j k \omega_0 t} \right)e^{-j \omega t} dt \\
      &= \frac{1}{T} \sum^{\infty}_{k = - \infty} \int^{\infty}_{-\infty} e^{j(\omega -k\omega_0)t} dt \\
      &= \frac{1}{T} \sum^{\infty}_{k = - \infty} \int^{\infty}_{-\infty} e^{j(\omega -k\omega_0)t} \cdot 1 dt \\
      &= \frac{1}{T} \sum^{\infty}_{k = - \infty} 2 \pi \delta(\omega - k \omega_0) \\
      &= \omega \sum^{\infty}_{k = - \infty} \delta(\omega - k \omega_0)
  \end{split}
\end{equation*}

ただし、4行目から5行目での変換には与えられたフーリエ変換表を用いている。
以上より、インパルス列のフーリエ変換もインパルス列となる。

\subsection*{(3)}
(3)で、$n(t) = 0$より$N(\omega) = 0$とすると
\begin{equation*}
  \begin{split}
    O(\omega) &= G(\omega) H(\omega) \\
    |O(\omega)| &= |G(\omega) H(\omega)| \\
    \log |O(\omega)| &= \log |G(\omega) H(\omega)| = \log |G(\omega)| + \log |H(\omega)| \\
  \end{split}
\end{equation*}

\subsection*{(4)}
(3)より

\[
  \log |H(\omega)| = \log |O(\omega)| - \log |G(\omega)| \\
\]
であり、$o(t)$から時刻$t_0$付近で$N$個の標本値を得るから、$g(t)$もインパルス列との畳み込みとして標本化される。\\
すると、(2)より、インパルス列のフーリエ変換はインパルス列となるから、$\log | G(\omega) |$もインパルス列となる。以上から、$\log | H(\omega) |$は標本化各周波数は$\omega_0$として$\log |O(\omega)|$の下の包絡線を取れば良い。\\
こうすると、$\log | O(\omega) |$の成分を取り除いたフィルタの特性が現れる。

\subsection*{(5)}
$o(t) = s(t) + n(t)$より

\[
  O(\omega, t) = S(\omega, t) + N(\omega, t)
\]

となります。したがって、振幅の二乗を考えれば

\begin{equation*}
  \begin{split}
    | O(\omega, t) |^2
      &= O(\omega) \overline{O(\omega)} \\
      &= (S(\omega, t) + N(\omega, t))\left(\overline{S(\omega, t)} + \overline{N(\omega, t)}\right) \\
      &= |S(\omega, t)|^2 + S(\omega, t) \overline{N(\omega, t)} + \overline{S(\omega, t)}N(\omega, t) + |N(\omega, t)|^2 \\
      &= |S(\omega, t)|^2 + 2Re(S(\omega, t) \overline{N(\omega, t)}) + |N(\omega, t)|^2 \\
  \end{split}
\end{equation*}

よって、
\begin{equation*}
  \begin{split}
    &\text{(5-1)} = |S(\omega, t)| \\
    &\text{(5-2)} = Re(S(\omega, t) \overline{N(\omega, t)})\\
    &\text{(5-3)} = |N(\omega, t)|
  \end{split}
\end{equation*}

$Re(S(\omega, t) \overline{N(\omega, t)})$の項は$S(\omega, t)$と$N(\omega, t)$の内積にあたるので、独立性より期待値は0に近づく。
よって、
\[
  \text{(5-4)} = Re(S(\omega, t) \overline{N(\omega, t)})
\]

また、
\[
  \text{(5-5)} = |O(\omega, t)|^2 - |N(\omega, t)|^2
\]

\end{document}

