% class
\documentclass[a4paper,12pt,xelatex,ja=standard]{bxjsarticle}

% packages
%% mathematical notations
\usepackage{amsthm,amsmath,amssymb,amsfonts} % mathematical notations
\usepackage{bm} % bold character
\usepackage{latexsym} % more mathematical notations
\usepackage{physics} % physical notations
%% graphs
\usepackage[dvipdfmx]{graphicx, xcolor} % graph
\usepackage{circuitikz} % for circuit elements
\usepackage{float} % positioning of graphs
\usepackage{siunitx} % SI units
\usepackage{tikz} % graphic elements
\usepackage{wrapfig} % must be after float package.
%% type system
\usepackage{bussproofs} % proof tree
%% code
\usepackage[ruled,vlined]{algorithm2e} % pseudo code
\usepackage{listings} % source code
\lstset{
  basicstyle=\footnotesize,
}
%% url
\usepackage{url}

% Basic information
\title{電子情報学専攻 \, 専門 \\ 平成25年 \, 解答・解説}
\author{diohabara}
\date{\today}

\begin{document}
\maketitle

\section*{第1問\ 電気・電子回路}
  \subsection*{(1)}
  \subsection*{(2)}
  \subsection*{(3)}
  \subsection*{(4)}
  \subsection*{(5)}

\section*{第2問\ コンピュータ・アーキテクチャ}
  \subsection*{(1)}
    \subsubsection*{空間局所性}
    あるメモリのデータが参照されたときに、そのデータの近くのアドレスが引き続き参照される性質。
    \subsubsection*{時間局所性}
    あるメモリのデータが参照されたときに、そのデータ短時間で再び参照される性質。
  \subsection*{(2)}
  ページテーブルとは主記憶上にある、仮想アドレスと物理アドレスの対応表である
  \subsection*{(3)}
  TLB、Translation Lookaside Bufferとはページテーブル専用のフルアソシアティブキャッシュである。高速にアドレス変換をするためにページテーブルの一部をTLB内に複製し変換する。TLB内に存在しないアドレスはページテーブルを参照する。
  \subsection*{(4)}
  \begin{enumerate}
    \item アドレス計算をする。
    \item TLBを参照する。TLBミスが起きる
    \item ページテーブルを参照する。ページフォールトが起きる。
    \item 例外を起こし、処理がOSに移る。
    \item OSが特権的命令を用いてページテーブル、主記憶、二次記憶の書き換えを行う。
    \item 命令実行を再開する。
  \end{enumerate}
  \subsection*{(5)}
  TLBミスの処理はキャッシュ、主記憶上で処理が起きハードウェアのみでも済むが、ページフォールトは割り込みが生じOSへ処理が移り特権命令を用いるなどソフトウェアでも処理することが前提だから。

\newpage
\section*{第3問\ アルゴリズム}
  \subsection*{(1)}
  \begin{lstlisting}[language=Python, caption=再帰呼び出し]
    def fib(n):
        if n <= 1:
            return n
        else:
            return fib(n-1) + fib(n-2)
  \end{lstlisting}
  \subsection*{(2)}
  \begin{lstlisting}[language=Python, caption=動的計画法]
    def fib(n):
        cache = [0] * n
        cache[1] = 1
        for i in range(2, n):
            cache[i] = cache[i-1] + cache[i-2]
        return cache[n-1]
  \end{lstlisting}
  \subsection*{(3)}
    \subsubsection*{(1)の問題点}
    \begin{itemize}
      \item 時間計算量が\(O(n^2)\)である点。\\
        再帰呼び出しの際に、2回再帰関数を呼び出している。そのため、\(n (> 1)\)で再帰関数を呼び出す回数が\(2^n\)となる。
      \item 再起呼び出しが深いことが原因で、スタックオーバーフローを起こす可能性がある。
      \item また、64ビット整数を用いて計算する際に答えがオーバーフローしてしまう可能性がある。
    \end{itemize}
    \subsubsection*{(2)の問題点}
    \begin{itemize}
      \item 空間計算量が\(O(n)\)である点。\\
        再帰呼び出しによる実装と比べて、時間計算量は\(O(n)\)になり改善されている。一方、配列を用いるため空間計算量は\(O(n)\)となり問題として残っている。
      \item また、64ビット整数を用いて計算する際に答えがオーバーフローしてしまう可能性がある。
    \end{itemize}
  \subsection*{(4)}
    \subsubsection*{一般項による計算の利点}
    空間計算量、時間計算量が\(O(1)\)であり(2)の方法よりも計算の効率が良い
    \subsubsection*{一般項による計算の欠点}
    浮動小数点数を用いて計算をするので誤差が生じる事があり、整数での計算となる(2)の方法よりも正確さに欠ける。

\section*{第4問\ ネットワーク}
  \subsection*{(1)}
  \subsection*{(2)}
  \subsection*{(3)}
  \subsection*{(4)}
  \subsection*{(5)}
    \subsubsection*{(a)}
    \subsubsection*{(b)}

\section*{第5問\ }
  \subsection*{(1)}
  \subsection*{(2)}
  \subsection*{(3)}
  \subsection*{(4)}
  \subsection*{(5)}
  \subsection*{(6)}
  \subsection*{(7)}

\end{document}

