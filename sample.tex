% ref
% https://students-tech.blog/post/tectonic.html#%E6%97%A9%E9%80%9F%E6%9B%B8%E3%81%84%E3%81%A6%E3%81%BF%E3%81%9F%E3%81%84

% class
\documentclass[a4paper,12pt,xelatex,ja=standard]{bxjsarticle}

% packages
%% mathematical notations
\usepackage{amsthm,amsmath,amssymb,amsfonts} % mathematical notations
\usepackage{bm} % bold character
\usepackage{latexsym} % more mathematical notations
\usepackage{physics} % physical notations
%% graphs
\usepackage[dvipdfmx]{graphicx, xcolor} % graph
\usepackage{circuitikz} % for circuit elements
\usepackage{float} % positioning of graphs
\usepackage{siunitx} % SI units
\usepackage{tikz} % graphic elements
\usepackage{wrapfig} % must be after float package.
%% type system
\usepackage{bussproofs} % proof tree
%% code
\usepackage[ruled,vlined]{algorithm2e} % pseudo code
\usepackage{listings} % source code

% Basic information
\title{xxx 年度解答・解説}
\author{作者}
\date{\today}

\begin{document}
\maketitle

\section{見出し}

X 年の問題は易化しました。

\subsection{回路図}

\begin{figure}[H]
  \begin{center}
    \begin{circuitikz}[american currents]
      \draw (0,0)
        to[sV=$E$] (0,2)
        to[short] (2,2)
        to[european resistor=$R$] (2,0)
        to[short] (0,0);
      \draw (2,2)
        to[short] (4,2)
        to[L=$L$] (4,0)
        to[short] (2,0);
      \draw (4,2)
        to[short] (6,2)
        to[C=$C$] (6,0)
        to[short] (4,0);
    \end{circuitikz}
    \caption{RLC 並列回路}
  \end{center}
\end{figure}

\subsection{証明木}
\begin{prooftree}
  \AxiomC{A}
  \AxiomC{B}
  \AxiomC{C}
  \TrinaryInfC{D}
\end{prooftree}

\subsection{擬似コード}
\begin{algorithm}[H]
\SetAlgoLined
\KwResult{Write here the result }
initialization\;
\While{While condition}{
  instructions\;
  \eIf{condition}{
  instructions1\;
  instructions2\;
  }{
  instructions3\;
  }
}
\caption{How to write algorithms}
\end{algorithm}

\subsection{ソースコード}
\begin{lstlisting}[language=c++]
#include <iostream>
int main(){
    std::cout << "Hello, world!" << std::endl;
    return 0;
}
\end{lstlisting}

\subsection*{なんか書いとけ!!!}
\[
  E = mc^2
\]

\end{document}

